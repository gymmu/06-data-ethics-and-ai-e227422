\documentclass{article}

\usepackage[ngerman]{babel}
\usepackage[utf8]{inputenc}
\usepackage[T1]{fontenc}
\usepackage{hyperref}
\usepackage{csquotes}

\usepackage[
    backend=biber,
    style=apa,
    sortlocale=de_DE,
    natbib=true,
    url=false,
    doi=false,
    sortcites=true,
    sorting=nyt,
    isbn=false,
    hyperref=true,
    backref=false,
    giveninits=false,
    eprint=false]{biblatex}
\addbibresource{../references/bibliography.bib}

\title{Review des Papers "Ethik im Umgang mit Daten" von Jaël Theiler}
\author{Aylina Ospina Cruz}
\date{03.06.2024}

\begin{document}
\maketitle

\abstract{
    Dies ist ein Review der Arbeit zum Thema Ethik im Umgang mit Daten von Jaël Theiler.Das Projekt hat sich mit dem Thema der digitalen Cookies und der Frage, ob diese ethisch korrekt sind, beschäftigt. 
}
\section{Positive Punkte}
Das Projekt ist gut in verschiedene Abschnitte unterteilt und es fällt einem so einfach, den Text zu lesen. Die Kapitel wurden ausserdem in sinnvolle Unterkapitel unterteil, was einen sehr ordentlichen Eindruck gibt. Die Funktion und das Trainieren der Künstlichen Intelligenz wurden sehr detailliert und verständlich erklärt.Man merkt, dass man sich gut mit dem Thema KI und Cookies auseinandergesetzt hat. Ausserdem wurde die Verknüpfung zwischen der künstlichen Intelligenz und den Cookies erfolgreich gemacht und der ganze Text hat einen guten Zusammenhang. Die Themen werden aufbauend erklärt und so ist es auch möglich, ohne Vorwissen den Text zu verstehen. Es wurden gut alle Problematiken bei den Cookies aufgezeigt und diese kritisch betrachtet. Ausserdem wurde am Schluss ein gutes Fazit gebildet. So kann der ganze Text nochmals gut zusammengefasst werden. Der gut strukturierte Text bietet eine gute Grundlage zum Verstehen dieser komplexen Themen und wurde immer aufbauend auf vorherige Punkte geschrieben. 

\section{Verbesserungsvorschläge}

Der Text ist grundsätzlich gut verständlich, es könnten jedoch Begriffe wie "neuronale Netze" ein wenig besser erklärt werden, da dies für nicht gut informierte Leser komplizierte und unverständliche Begriffe sind. Ausserdem können konkrete Beispiele zu den einzelenen erklärten Punkten das Verständniss fördern. Ein weiterer Vorschlag wäre, Zitate oder Quellen im Text zu integrieren, da dies das Interesse der Leser steigert. Die Zusammenfassung könnte man ein wenig länger machen und die verschiedenen Fragen ein wenig ausführlicher schreiben. So wären die verschiedenen Fragestellungen einfacher zu lesen und die Neugier der Leser würde geweckt werden.
\printbibliography

\end{document}
