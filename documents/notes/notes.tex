\documentclass{article}

\usepackage[ngerman]{babel}
\usepackage[utf8]{inputenc}
\usepackage[T1]{fontenc}
\usepackage{hyperref}
\usepackage{csquotes}

\usepackage[
    backend=biber,
    style=apa,
    sortlocale=de_DE,
    natbib=true,
    url=false,
    doi=false,
    sortcites=true,
    sorting=nyt,
    isbn=false,
    hyperref=true,
    backref=false,
    giveninits=false,
    eprint=false]{biblatex}
\addbibresource{../references/bibliography.bib}

\title{Notizen zum Projekt Data Ethics}
\author{Aylina Ospina Cruz}
\date{\today}

\begin{document}
\maketitle

\abstract{
    Inwiefern beeinflusst von KI erstellte personalisierte Webung das kaufverhalten der Menschen.
    Algorithmen und Datenanalyse: KI-Algorithmen analysieren das Verhalten und die Vorlieben von Nutzern, um personalisierte Produktempfehlungen zu geben. Beispiele sind Amazon und Netflix, die personalisierte Kauf- und Sehempfehlungen basierend auf bisherigen Käufen und angesehenen Inhalten bieten.
}

\tableofcontents


\section{Einleitung}

\printbibliography

\end{document}

