\documentclass{article}

\usepackage[ngerman]{babel}
\usepackage[utf8]{inputenc}
\usepackage[T1]{fontenc}
\usepackage{hyperref}
\usepackage{csquotes}

\usepackage[
    backend=biber,
    style=apa,
    sortlocale=de_DE,
    natbib=true,
    url=false,
    doi=false,
    sortcites=true,
    sorting=nyt,
    isbn=false,
    hyperref=true,
    backref=false,
    giveninits=false,
    eprint=false]{biblatex}
\addbibresource{../references/bibliography.bib}

\title{Notizen zum Projekt Data Ethics}
\author{Aylina Ospina Cruz}
\date{\today}

\begin{document}
\maketitle

\abstract{
    Inwiefern beeinflusst von KI erstellte personalisierte Webung das kaufverhalten der Menschen.
    Algorithmen und Datenanalyse: KI-Algorithmen analysieren das Verhalten und die Vorlieben von Nutzern, um personalisierte Produktempfehlungen zu geben. Beispiele sind Amazon und Netflix, die personalisierte Kauf- und Sehempfehlungen basierend auf bisherigen Käufen und angesehenen Inhalten bieten.
}

\tableofcontents


\section{Künsliche Inteligenz}

Die künstliche Inteligenz ermöglicht Maschinen und Computer Aufgaben zu erledigen, die eigentlich menschliche Inteligenz erfordern. Dies ist möglich, in dem man grosse Mengen an Daten sammelt und analysiert. Diese werden dann nach wiederkehrenden Mustern abgesucht damit so Modelle erstellt werden können. Die künstliche Inteligenz wird danach mit diesen Daten trainiert und ist so imstande, Sachen wahrzunemen und zu entscheiden. Sie wird immer wieder korrigiert und optimiert ihre Antworten. 

\section{personalisierte Werbung}
 Die Werbung unterscheidet sich von Person zu Person. Es werden passgenaue Anzeigen erstellt, die die Vorleiben, Interessen und Kaufverhalten beachten und so angepasste, personalisierte Werbung zuscheidet. Dies ist nur Möglich, wenn die Daten der Käufer gesammelt und anayilisert werden. Somit ist es Möglich, Werbung genauen Zielgruppen zu zeigen und so hat man eine viel grössere Erfolgquote. Die Firmen können genau definieren, wen sie erreichen wollen und können auch verschiedene Werbung erstellen, die einige Menschen mehr anspricht wie andere. Um personalisierte Werbung zu ermöglichen, werden häufig Tracking-Technologien wie Cookies eingesetzt. Diese helfen, unser Online-Verhalten zu verfolgen und Informationen über unsere Präferenzen und Vorlieben zu sammeln. 

 \section{KI in der personalisierten Werbung}

 Die künstliche Intelligenz spielt eine grosse Rolle für die personalisierte Werbung. Sie funktioniert mithilfe von Machine Learning-Algorithmen, die den passenden Zeitpunkt und die passende Plattform für die Werbekampagne bestimmt. Machine Learning-Algorithmen sind darauf trainiert, Muster in großen Beständen historischer Daten zu finden und auf Basis dieser Erkenntnissen und Kommentaren, Vorhersagen zu treffen. Sie helfen dabei, personalisierte Werbung zu schalten, die passende Plattform und den richtigen Zeitpunkt für Werbekampagnen zu bestimmen. 

 \section{Einfluss auf das kaufverhalten}
 Da die Werbung auf den Käufer angepasst wird, 


\printbibliography

\end{document}

